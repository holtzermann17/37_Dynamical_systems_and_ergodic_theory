\documentclass[12pt]{article}
\usepackage{pmmeta}
\pmcanonicalname{HyperbolicSet}
\pmcreated{2013-03-22 13:40:21}
\pmmodified{2013-03-22 13:40:21}
\pmowner{Koro}{127}
\pmmodifier{Koro}{127}
\pmtitle{hyperbolic set}
\pmrecord{5}{34338}
\pmprivacy{1}
\pmauthor{Koro}{127}
\pmtype{Definition}
\pmcomment{trigger rebuild}
\pmclassification{msc}{37D20}
\pmsynonym{hyperbolic structure}{HyperbolicSet}
\pmsynonym{uniformly hyperbolic}{HyperbolicSet}
\pmrelated{HyperbolicFixedPoint}

% this is the default PlanetMath preamble.  as your knowledge
% of TeX increases, you will probably want to edit this, but
% it should be fine as is for beginners.

% almost certainly you want these
\usepackage{amssymb}
\usepackage{amsmath}
\usepackage{amsfonts}
\usepackage{mathrsfs}

% used for TeXing text within eps files
%\usepackage{psfrag}
% need this for including graphics (\includegraphics)
%\usepackage{graphicx}
% for neatly defining theorems and propositions
%\usepackage{amsthm}
% making logically defined graphics
%%%\usepackage{xypic}

% there are many more packages, add them here as you need them

% define commands here
\newcommand{\Cdiff}{\mathcal{C}}
\newcommand{\C}{\mathbb{C}}
\newcommand{\R}{\mathbb{R}}
\newcommand{\N}{\mathbb{N}}
\newcommand{\Z}{\mathbb{Z}}
\begin{document}
Let $M$ be a compact smooth manifold, and let $f:M\to M$ be a diffeomorphism.
An $f$-invariant subset $\Lambda$ of $M$ is said to be \emph{hyperbolic} (or to have an hyperbolic structure) if there is a splitting of the tangent bundle of $M$ restricted to $\Lambda$ into a (Whitney) sum of two $Df$-invariant subbundles, $E^s$ and $E^u$ such that the restriction of $Df|_{E^s}$ is a contraction and $Df|_{E^u}$ is an expansion. This means that there are constants $0<\lambda<1$ and $c>0$ such that
\begin{enumerate}
\item $T_\Lambda M = E^s\oplus E^u$;
\item $Df(x)E^s_x = E^s_{f(x)}$ and $Df(x)E^u_x = E^u_{f(x)}$ for each $x\in \Lambda$;
\item $\|Df^nv\| < c\lambda^n\|v\|$ for each $v\in E^s$ and $n> 0$;
\item $\|Df^{-n}v\| < c\lambda^n \|v\|$ for each $v\in E^u$ and $n>0$.
\end{enumerate}
using some Riemannian metric on $M$.

If $\Lambda$ is hyperbolic, then there exists an \emph{adapted} Riemannian metric, i.e. one such that $c=1$.

%\textbf{Remark}: Morse-Smale diffeomorphisms have very good behaviour, so the 
%dynamics of a Morse-Smale system (i.e. the behaviour of the orbits of 
%elements of $M$ when $f$ is applied to them) is reasonably easy to describe. 
%Since the nonwandering set of such a diffeomorphism consists only of periodic 
%points (and finitely many of them), every other orbit converges to one of 
%those periodic orbits. The notion of hyperbolic sets allow us to (somehow) 
%generalize these ideas for a much wider class of diffeomorphisms (namely, 
%those satisfying Axiom A) by means Smale's spectral decomposition theorem.

% Furthermore: the stable manifold theorem for hyperbolic sets says that
% the local stable and unstable manifold of a point in a hyperbolic set are 
% always $\Cdiff^r$-embedded disks, and the (global) stable and unstable 
% manifolds are (injectively) $C^k$-immersed disks in $M$. 
%Also, if $\Lambda$ 
% has local product structure (or, equivalently, if it is locally maximal), the
% stable (likewise unstable) manifolds form a foliation of $M$.
% Some other notions from the fixed point theory can be generalized as well.

%The standard reference for this topic is \cite{Smale}; but it is covered in 
%many differentiable dynamical systems books, for example \cite{Shub}.

%\begin{thebibliography}{9}

%\bibitem{Shub}
%Shub, M. 
%\emph{Global Stability of Dynamical Systems}, New York, Springer-Verlag, 1987. 

%\bibitem{Smale}
%Smale, S.
%\emph{Differnetiable dynamical systems},
%Bull. Amer. Math. Soc. \textbf{73} (1967), 747-817.
%\end{thebibliography}
%%%%%
%%%%%
\end{document}
