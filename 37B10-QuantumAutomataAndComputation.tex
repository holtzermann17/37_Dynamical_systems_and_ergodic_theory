\documentclass[12pt]{article}
\usepackage{pmmeta}
\pmcanonicalname{QuantumAutomataAndComputation}
\pmcreated{2013-03-22 18:10:17}
\pmmodified{2013-03-22 18:10:17}
\pmowner{bci1}{20947}
\pmmodifier{bci1}{20947}
\pmtitle{quantum automata and computation}
\pmrecord{172}{40734}
\pmprivacy{1}
\pmauthor{bci1}{20947}
\pmtype{Topic}
\pmcomment{trigger rebuild}
\pmclassification{msc}{37B10}
\pmclassification{msc}{55U35}
\pmclassification{msc}{03D15}
\pmclassification{msc}{03D05}
\pmsynonym{nano-automata}{QuantumAutomataAndComputation}
%\pmkeywords{quantum computers}
%\pmkeywords{quantum groupoids}
%\pmkeywords{quantum groups}
%\pmkeywords{quantum applications}
%\pmkeywords{quantum relational biology}
\pmrelated{Automata}
\pmrelated{QuantumNanoAutomata}
\pmrelated{QuantumGroupoids2}
\pmrelated{GroupoidCDynamicalSystem}
\pmrelated{CategoryOfAutomata}
\pmrelated{CategoryOfQuantumAutomata}
\pmrelated{TopicEntryOnFoundationsOfMathematics}
\pmrelated{QuantumSpaceTimes}
\pmrelated{ETAS}
\pmrelated{QuantumCategory}
\pmrelated{TopicEntryOnAppliedMathematics}
\pmrelated{RichardFeynman}
\pmrelated{Exam}
\pmdefines{quantum automaton}
\pmdefines{quantum computers}

% define commands here

\usepackage{amsmath, amssymb, amsfonts, amsthm, amscd, latexsym,color,enumerate}
%%\usepackage{xypic}
\xyoption{curve}
\usepackage[mathscr]{eucal}

\setlength{\textwidth}{6.5in}
%\setlength{\textwidth}{16cm}
\setlength{\textheight}{9.0in}
%\setlength{\textheight}{24cm}

\hoffset=-.75in     %%ps format
%\hoffset=-1.0in     %%hp format
\voffset=-.4in

%the next gives two direction arrows at the top of a 2 x 2 matrix

\newcommand{\directs}[2]{\def\objectstyle{\scriptstyle}  \objectmargin={0pt}
\xy
(0,4)*+{}="a",(0,-2)*+{\rule{0em}{1.5ex}#2}="b",(7,4)*+{\;#1}="c"
\ar@{->} "a";"b" \ar @{->}"a";"c" \endxy }

\theoremstyle{plain}
\newtheorem{lemma}{Lemma}[section]
\newtheorem{proposition}{Proposition}[section]
\newtheorem{theorem}{Theorem}[section]
\newtheorem{corollary}{Corollary}[section]
\newtheorem{conjecture}{Conjecture}[section]

\theoremstyle{definition}
\newtheorem{definition}{Definition}[section]
\newtheorem{example}{Example}[section]
%\theoremstyle{remark}
\newtheorem{remark}{Remark}[section]
\newtheorem*{notation}{Notation}
\newtheorem*{claim}{Claim}


\theoremstyle{plain}
\renewcommand{\thefootnote}{\ensuremath{\fnsymbol{footnote}}}
\numberwithin{equation}{section}
\newcommand{\Ad}{{\rm Ad}}
\newcommand{\Aut}{{\rm Aut}}
\newcommand{\Cl}{{\rm Cl}}
\newcommand{\Co}{{\rm Co}}
\newcommand{\DES}{{\rm DES}}
\newcommand{\Diff}{{\rm Diff}}
\newcommand{\Dom}{{\rm Dom}}
\newcommand{\Hol}{{\rm Hol}}
\newcommand{\Mon}{{\rm Mon}}
\newcommand{\Hom}{{\rm Hom}}
\newcommand{\Ker}{{\rm Ker}}
\newcommand{\Ind}{{\rm Ind}}
\newcommand{\IM}{{\rm Im}}
\newcommand{\Is}{{\rm Is}}
\newcommand{\ID}{{\rm id}}
\newcommand{\GL}{{\rm GL}}
\newcommand{\Iso}{{\rm Iso}}
\newcommand{\Sem}{{\rm Sem}}
\newcommand{\St}{{\rm St}}
\newcommand{\Sym}{{\rm Sym}}
\newcommand{\SU}{{\rm SU}}
\newcommand{\Tor}{{\rm Tor}}
\newcommand{\U}{{\rm U}}

\newcommand{\A}{\mathcal A}
\newcommand{\D}{\mathcal D}
\newcommand{\E}{\mathcal E}
\newcommand{\F}{\mathcal F}
\newcommand{\G}{\mathcal G}
\newcommand{\R}{\mathcal R}
\newcommand{\cS}{\mathcal S}
\newcommand{\cU}{\mathcal U}
\newcommand{\W}{\mathcal W}

\newcommand{\Ce}{\mathsf{C}}
\newcommand{\Q}{\mathsf{Q}}
\newcommand{\grp}{\mathsf{G}}
\newcommand{\dgrp}{\mathsf{D}}

\newcommand{\bA}{\mathbb{A}}
\newcommand{\bB}{\mathbb{B}}
\newcommand{\bC}{\mathbb{C}}
\newcommand{\bD}{\mathbb{D}}
\newcommand{\bE}{\mathbb{E}}
\newcommand{\bF}{\mathbb{F}}
\newcommand{\bG}{\mathbb{G}}
\newcommand{\bK}{\mathbb{K}}
\newcommand{\bM}{\mathbb{M}}
\newcommand{\bN}{\mathbb{N}}
\newcommand{\bO}{\mathbb{O}}
\newcommand{\bP}{\mathbb{P}}
\newcommand{\bR}{\mathbb{R}}
\newcommand{\bV}{\mathbb{V}}
\newcommand{\bZ}{\mathbb{Z}}

\newcommand{\bfE}{\mathbf{E}}
\newcommand{\bfX}{\mathbf{X}}
\newcommand{\bfY}{\mathbf{Y}}
\newcommand{\bfZ}{\mathbf{Z}}

\renewcommand{\O}{\Omega}
\renewcommand{\o}{\omega}
\newcommand{\vp}{\varphi}
\newcommand{\vep}{\varepsilon}

\newcommand{\diag}{{\rm diag}}
\newcommand{\desp}{{\mathbb D^{\rm{es}}}}
\newcommand{\Geod}{{\rm Geod}}
\newcommand{\geod}{{\rm geod}}
\newcommand{\hgr}{{\mathbb H}}
\newcommand{\mgr}{{\mathbb M}}
\newcommand{\ob}{\operatorname{Ob}}
\newcommand{\obg}{{\rm Ob(\mathbb G)}}
\newcommand{\obgp}{{\rm Ob(\mathbb G')}}
\newcommand{\obh}{{\rm Ob(\mathbb H)}}
\newcommand{\Osmooth}{{\Omega^{\infty}(X,*)}}
\newcommand{\ghomotop}{{\rho_2^{\square}}}
\newcommand{\gcalp}{{\mathbb G(\mathcal P)}}
\newcommand{\rf}{{R_{\mathcal F}}}
\newcommand{\glob}{{\rm glob}}
\newcommand{\loc}{{\rm loc}}
\newcommand{\TOP}{{\rm TOP}}

\newcommand{\wti}{\widetilde}
\newcommand{\what}{\widehat}

\renewcommand{\a}{\alpha}
\newcommand{\be}{\beta}
\newcommand{\ga}{\gamma}
\newcommand{\Ga}{\Gamma}
\newcommand{\de}{\delta}
\newcommand{\del}{\partial}
\newcommand{\ka}{\kappa}
\newcommand{\si}{\sigma}
\newcommand{\ta}{\tau}
\newcommand{\lra}{{\longrightarrow}}
\newcommand{\ra}{{\rightarrow}}
\newcommand{\rat}{{\rightarrowtail}}
\newcommand{\oset}[1]{\overset {#1}{\ra}}
\newcommand{\osetl}[1]{\overset {#1}{\lra}}
\newcommand{\hr}{{\hookrightarrow}}
\newcommand{\hdgb}{\boldsymbol{\rho}^\square}
\newcommand{\hdg}{\rho^\square_2}
\newcommand{\med}{\medbreak}
\newcommand{\medn}{\medbreak \noindent}
\newcommand{\bign}{\bigbreak \noindent}

\renewcommand{\leq}{{\leqslant}}
\renewcommand{\geq}{{\geqslant}}
\def\red{\textcolor{red}}
\def\magenta{\textcolor{magenta}}
\def\blue{\textcolor{blue}}
\def\<{\langle}
\def\>{\rangle}

\begin{document}
\section{Topic on Quantum Automata and Quantum Computations}

\subsection{Introduction}
 A quantum automaton can be simply described as an extension of an automaton with quantum states instead of the sequentially determined states, inputs and outputs of a \textit{sequential, or state machine}.
The precise mathematical definitions of quantum automaton, variable automaton, and quantum computation were first introduced formally in refs. \cite{ICB71a} and \cite{ICB71b} in relation to relational models in Quantum Relational Biology (\textit{loc.cit}).


\subsection{Historical Note}  

 Quantum computation and quantum machines (or nanobots) were much publicized in the early 1980's by Richard Feynman (Nobel Laureate in Physics: QED),and, subsequently, a very large number of papers- too many to cite all of them here- were published on this topic by a rapidly growing number of quantum theoreticians and some applied mathematicians.

\subsection{Quantum Automata}

\begin{definition} A simple definition of {\em quantum automaton} is obtained by considering instead of the transition function of a classical sequential machine, the (quantum) transitions in a finite quantum system with definite probabilities determined by quantum dynamics. The {\em quantum state space} of a {\em quantum automaton} is thus defined as a quantum groupoid over a bundle of Hilbert spaces, or over rigged Hilbert spaces. Formally, whereas a sequential machine, or state machine with state space S, input set I and output set O, is defined as a quintuple: $(S, I, O, \delta : S  \times  S \rightarrow  S, \lambda: S \times I\rightarrow  O)$, a quantum automaton is defined by a triple $(\emph{H}, \Delta: \emph{H}  \rightarrow \emph{H}, \mu)$, where \emph{H} is either a Hilbert space or a rigged Hilbert space of quantum states and operators acting on \emph{H}, and $\mu$ is a measure related to the quantum logic, LM, and (quantum) transition probabilities of this quantum system. 
\end{definition} 

\begin{remark} 
 Quantum computation becomes possible only when macroscopic blocks of quantum states can be controlled {\em via} quantum preparation and subsequent, classical observation. Obstructions to building, or constructing quantum computers are known to exist in dimensions greater than $2$ as a result of the standard K-S theorem. Subsequent definitions of quantum computers reflect attempts to either avoid or surmount such difficulties often without seeking solutions through quantum operator algebras and their representations related to extended quantum symmetries which define fundamental invariants that are key to realizing quantum computation.
\end{remark} 

\begin{definition}: alternatively, as a quantum algebraic topology object, a \emph{quantum automaton} is defined by the
triplet $(\grp,\emph{H}-\Re_{\grp}, Aut(\grp)$), where $\grp$ is a locally compact \emph{quantum groupoid}, 
\emph{H}-$\Re_{\grp}$ are the unitary representations of $\grp$ on rigged Hilbert spaces $\Re_\grp$ of quantum states and quantum operators on \emph{H}, and $Aut(\grp)$ is the transformation, or automorphism, groupoid of quantum transitions.
\end{definition}

\textbf{Remark.} Other definitions of quantum automata and quantum computations have also been reported that are
closely related to recent experimental attempts at constructing quantum computing devices. 

 Two examples of such definitions are briefly considered next.

\begin{definition}:  {\em quantum automata} were defined in refs.\cite{ICB71a} and \cite{ICB71b} as generalized, probabilistic automata with quantum state spaces. Their next-state functions operate through transitions between quantum states defined by the quantum equations of motions in the Schr\"{o}dinger representation, with both initial and boundary conditions in space-time.

A new theorem was proven which states that the \emph{category of quantum automata and automata--homomorphisms has both
limits and colimits.} Therefore, both categories of quantum automata and classical automata (sequential machines) are
\emph{bicomplete.} A second new theorem established that the standard automata category is a subcategory of the quantum
automata category. 
\end{definition}

\subsection{Quantum Automata Applications to Modeling Complex Systems}
The quantum automata category has a faithful representation in the category of Generalized 
$(M,R)$ -systems which are open, dynamic bio-networks (\cite{ICB87}) with defined biological relations that represent physiological functions of primordial(s), single cells and the simpler organisms. A new \emph{category of quantum computers} is also defined in terms of \emph{reversible} quantum automata with quantum state spaces represented by topological groupoids that admit a local characterization through unique `quantum' \emph{Lie algebroids}. On the other hand, the category of $n$-\L ukasiewicz algebras has a subcategory of \emph{centered} $n$- \L{}ukasiewicz algebras \cite{GV70}  (which can be employed to design and construct subcategories of quantum automata based on $n$-{}\L ukasiewicz diagrams of existing VLSI. Furthermore, as shown in ref.(\cite{GV70} the category of centered $n$-{}\L ukasiewicz algebras and the category of Boolean algebras are naturally equivalent. 
\med
  Variable machines with a varying transition function were previously discussed informally by Norbert Wiener as a possible model for complex biological systems although how this might be achieved in \textit{Biocybernetics} has not been specifcally, or mathematically presented by Wiener. 
\med
  A `no-go' conjecture was also proposed which states that Generalized (\textbf{M,R})--Systems complexity prevents their complete computability by either standard or quantum automata. The concepts of quantum automata and quantum computation were initially studied and are also currently further investigated in the contexts of quantum genetics, genetic networks with nonlinear dynamics, and bioinformatics. In a previous publication (ICB71a)-- after introducing the formal concept of quantum automaton--the possible implications of this concept for correctly modeling genetic and metabolic activities in living cells and organisms were also considered. This was followed by a formal report on quantum and abstract, symbolic computation based on the theory of categories, functors and natural transformations \cite{ICB71b}. The notions of topological semigroup, quantum automaton,or quantum computer, were then suggested with a view to their potential applications to the analogous simulation of biological systems, and especially genetic activities and nonlinear dynamics in genetic networks. Further, detailed studies of nonlinear
dynamics in genetic networks were carried out in categories of $n$-valued, \L ukasiewicz Logic
Algebras that showed significant dissimilarities \cite{ICB77} from the widespread Bolean models of human
neural networks that may have begun with the early publication of \cite{MP45}. Molecular models in terms of categories, functors and natural transformations were then formulated for uni-molecular chemical transformations, multi-molecular chemical and biochemical transformations \cite{ICB2k4a}. Previous applications of computer modeling, classical automata theory, and relational biology to molecular biology, oncogenesis and medicine were extensively reviewed and several important conclusions were reached regarding both the potential and limitations of the computation-assisted modeling of
biological systems, and especially complex organisms such as \emph{Homo sapiens sapiens} \cite{ICB87}. Novel approaches to solving the realization problems of Relational Biology models in Complex System Biology are introduced in terms of natural transformations between functors of such molecular categories. Several applications of such natural transformations of functors were then presented to protein biosynthesis, embryogenesis and nuclear transplant
experiments. Other possible realizations in Molecular Biology and Relational Biology of
Organisms were then suggested in terms of quantum automata models of Quantum Genetics and
Interactomics. Future developments of this novel approach are likely to also include: Fuzzy
Relations in Biology and Epigenomics, Relational Biology modeling of Complex Immunological
and Hormonal regulatory systems, $n$-categories and \emph{generalized $LM$}--Topoi of 
\L{}ukasiewicz Logic Algebras and intuitionistic logic (Heyting) algebras for modeling nonlinear dynamics and cognitive processes in complex neural networks that are present in the human brain, as well as stochastic modeling of
genetic networks in \L{}ukasiewicz Logic Algebras (LLA). 

\subsubsection{Quantum Automata,Computation and Dynamics Represented by Categories and Functors} 

Molecular models were previously defined in terms of categories, functors and natural transformations were
formulated for unimolecular chemical transformations, multi-molecular chemical and biochemical transformations 
\cite{ICB2k4}. Dynamic similarities or analogies between categories of classical, quantum or complex systems and their transformations were then naturally represented in terms of adjoint functors and the corresponding natural equivalences.  

\textbf{Remark}. Previous applications of computer modeling, classical automata theory, and relational biology to molecular biology, neural networks, oncogenesis and medicine were extensively reviewed in a previous monograph and several important conclusions were reached regarding both the potential and the severe limitations of the algorithm driven, recursive computation-assisted modeling of complex biological systems \cite{ICB87}.


\begin{thebibliography}{9}

\bibitem{ICB71a}
Baianu, I.C.: 1971a, Categories, Functors and Quantum Algebraic Computations, in P. Suppes (ed.), \emph{Proceed. Fourth Intl. Congress Logic-Mathematics-Philosophy of Science}, September 1--4, 1971, the University of Bucharest.

\bibitem{ICB71b}
Baianu, I.C.: 1971b, Organismic Supercategories and Qualitative Dynamics of Systems. \emph{Bulletin of Mathematical Biophysics}, \textbf{33} (3): 339--354.

\bibitem{ICB87}
Baianu, I.C. 1987. Computer Models and Automata Theory in Biology and Medicine. (A Review). In:
\emph{``Mathematical Models in Medicine."},vol.7., M. Witten, Ed., Pergamon Press: New York,
 pp.1513-1577.

\bibitem{ICB04b}
I.C. Baianu.: \L ukasiewicz-Topos Models of Neural Networks, Cell Genome and Interactome Nonlinear Dynamics). CERN Preprint EXT-2004-059. \textit{Health Physics and Radiation Effects} (June 29, 2004). 

\bibitem{BGB06}
I. C. Baianu, J. F. Glazebrook, R. Brown and G. Georgescu.: Complex Nonlinear Biodynamics in Categories, Higher dimensional Algebra and \L ukasiewicz-Moisil Topos: Transformation of Neural, Genetic and Neoplastic 
Networks, \emph{Axiomathes}, \textbf{16}: 65--122(2006).

\bibitem{ICB3}
Baianu, I.C.: 1970, Organismic Supercategories: II. On Multistable Systems. \emph{Bulletin of Mathematical Biophysics}, \textbf{32}: 539-561.

\bibitem{ICBS73}
Baianu, I.C. and D. Scripcariu: 1973, On Adjoint Dynamical Systems. \emph{The Bulletin of Mathematical Biophysics}, \textbf{35}(4), 475--486.

\bibitem{ICB5}
Baianu, I.C.: 1973, Some Algebraic Properties of \emph{\textbf{(M,R)}} -- Systems. \emph{Bulletin of Mathematical Biophysics} \textbf{35}, 213-217.

\bibitem{ICBm2}
Baianu, I.C. and M. Marinescu: 1974, A Functorial Construction of \emph{\textbf{(M,R)}}-- Systems. \emph{Revue Roumaine de Mathematiques Pures et Appliquees} \textbf{19}: 388-391.

\bibitem{ICB6}
Baianu, I.C.: 1977, A Logical Model of Genetic Activities in \L ukasiewicz
Algebras: The Non-linear Theory. \emph{Bulletin of Mathematical Biophysics},
\textbf{39}: 249-258.

\bibitem{ICB2}
Baianu, I.C.: 1980, Natural Transformations of Organismic Structures. \emph{Bulletin of Mathematical Biophysics}
\textbf{42}: 431-446


\bibitem{ICB2}
Baianu, I. C.: 1987a, Computer Models and Automata Theory in Biology and Medicine.,  in M. Witten 
(ed.), \emph{Mathematical Models in Medicine}, vol. 7., Pergamon Press, New York, 1513--1577; \PMlinkexternal{CERN Preprint No. EXT-2004-072}{http://doe.cern.ch//archive/electronic/other/ext/ext-2004-072.pdf} .

\bibitem{ICB2k4}
Baianu, I.C.: 2004, Quantum Nano--Automata (QNA): Microphysical Measurements with Microphysical QNA Instruments, \PMlinkexternal{CERN Preprint EXT--2004--125}{http://documents.cern.ch/cgi-bin/setlink?base=preprint&categ=ext&id=ext-2004-125}.


\bibitem{Bgg2}
Baianu, I. C., Glazebrook, J. F. and G. Georgescu: 2004,
Categories of Quantum Automata and N-Valued \L ukasiewicz Algebras
in Relation to Dynamic Bionetworks, \textbf{(M,R)}--Systems and
Their Higher Dimensional Algebra, \PMlinkexternal{Abstract and Preprint of
Report}{http://www.ag.uiuc.edu/fs401/QAuto.pdf} 

\bibitem{BBGG1}
Baianu I. C., Brown R., Georgescu G. and J. F. Glazebrook: 2006,
Complex Nonlinear Biodynamics in Categories, Higher Dimensional
Algebra and \L ukasiewicz--Moisil Topos: Transformations of
Neuronal, Genetic and Neoplastic networks, \emph{Axiomathes}
\textbf{16} Nos. 1--2, 65--122.

\bibitem{GV70}
Georgescu, G. and C. Vraciu 1970. On the Characterization of \L ukasiewicz Algebras., \emph{J Algebra}, 16 (4), 486-495.

\bibitem{BMB} 
BMB1:
\PMlinkexternal{Mathematical Biology reports}{http://www.kli.ac.at/theorylab/Keyword/R/RelationalBio.html}

\bibitem{MP45}
McCullough, E. and M. Pitts.1945. {\em Bull. Math. Biophys}. 7, 112-145.

\bibitem{Relevant URLs} 
MBR2: Eprint at \emph{cogprints.org/3674/}
\PMlinkexternal{Mathematical Biology reports2}{http://cogprints.org/3674/}.

\end{thebibliography}

%%%%%
%%%%%
\end{document}
