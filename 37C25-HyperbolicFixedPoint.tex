\documentclass[12pt]{article}
\usepackage{pmmeta}
\pmcanonicalname{HyperbolicFixedPoint}
\pmcreated{2013-03-22 13:47:57}
\pmmodified{2013-03-22 13:47:57}
\pmowner{Koro}{127}
\pmmodifier{Koro}{127}
\pmtitle{hyperbolic fixed point}
\pmrecord{6}{34516}
\pmprivacy{1}
\pmauthor{Koro}{127}
\pmtype{Definition}
\pmcomment{trigger rebuild}
\pmclassification{msc}{37C25}
\pmclassification{msc}{37D05}
\pmrelated{StableManifold}
\pmrelated{HyperbolicSet}
\pmdefines{hyperbolic periodic point}
\pmdefines{source}
\pmdefines{sink}
\pmdefines{saddle}

\endmetadata

% this is the default PlanetMath preamble.  as your knowledge
% of TeX increases, you will probably want to edit this, but
% it should be fine as is for beginners.

% almost certainly you want these
\usepackage{amssymb}
\usepackage{amsmath}
\usepackage{amsfonts}
\usepackage{mathrsfs}

% used for TeXing text within eps files
%\usepackage{psfrag}
% need this for including graphics (\includegraphics)
%\usepackage{graphicx}
% for neatly defining theorems and propositions
%\usepackage{amsthm}
% making logically defined graphics
%%%\usepackage{xypic}

% there are many more packages, add them here as you need them

% define commands here
\newcommand{\C}{\mathbb{C}}
\newcommand{\R}{\mathbb{R}}
\newcommand{\N}{\mathbb{N}}
\newcommand{\Z}{\mathbb{Z}}
\newcommand{\Per}{\operatorname{Per}}
\begin{document}
Let $M$ be a smooth manifold. A fixed point $x$ of a diffeomorphism 
$f\colon M\to M$ is said to be a \textbf{hyperbolic fixed point} if $Df(x)$ is a linear hyperbolic isomorphism. If $x$ is a periodic point of least period $n$, it is called a \textbf{hyperbolic periodic point} if it is a hyperbolic fixed point of $f^n$ (the $n$-th iterate of $f$).

If the dimension of the stable manifold of a fixed point is zero, the point is called a \textbf{source}; if the dimension of its unstable manifold is zero, it is called a \textbf{sink}; and if both the stable and unstable manifold have nonzero dimension, it is called a \textbf{saddle}.
%%%%%
%%%%%
\end{document}
