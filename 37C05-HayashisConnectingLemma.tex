\documentclass[12pt]{article}
\usepackage{pmmeta}
\pmcanonicalname{HayashisConnectingLemma}
\pmcreated{2013-03-22 14:07:16}
\pmmodified{2013-03-22 14:07:16}
\pmowner{Koro}{127}
\pmmodifier{Koro}{127}
\pmtitle{Hayashi's connecting lemma}
\pmrecord{6}{35527}
\pmprivacy{1}
\pmauthor{Koro}{127}
\pmtype{Theorem}
\pmcomment{trigger rebuild}
\pmclassification{msc}{37C05}
\pmclassification{msc}{37C25}
\pmsynonym{connecting lemma}{HayashisConnectingLemma}

% this is the default PlanetMath preamble.  as your knowledge
% of TeX increases, you will probably want to edit this, but
% it should be fine as is for beginners.

% almost certainly you want these
\usepackage{amssymb}
\usepackage{amsmath}
\usepackage{amsfonts}
\usepackage{mathrsfs}

% used for TeXing text within eps files
%\usepackage{psfrag}
% need this for including graphics (\includegraphics)
%\usepackage{graphicx}
% for neatly defining theorems and propositions
%\usepackage{amsthm}
% making logically defined graphics
%%%\usepackage{xypic}

% there are many more packages, add them here as you need them

% define commands here
\newcommand{\C}{\mathbb{C}}
\newcommand{\R}{\mathbb{R}}
\newcommand{\N}{\mathbb{N}}
\newcommand{\Z}{\mathbb{Z}}
\newcommand{\Per}{\operatorname{Per}}
\begin{document}
Let $f\colon M\to M$ be a $C^1$ diffeomorphism of the compact smooth manifold $M$, and let $p,q\in M$ be such that there exists a nonperiodic point in $\omega(p,f)\cap \alpha(q,f)$ (the intersection of the alpha limit set of $q$ with the omega limit set of $p$). Then there exists a diffeomorphism $g$, arbitrarily close to $f$ in the $\mathcal{C}^1$ topology of $\operatorname{Diff}^1(M)$, such that $q$ is in the forward orbit of $p$ through $g$, i.e. such that $g^n(p)=q$ for some $n>0$.

\begin{thebibliography}{9}
\bibitem{WenXia} Wen, L., Xia, Z., \emph{$\mathcal{C}^1$ connecting lemmas}, Trans. Amer. Math. Soc. \textbf{352} (2000), no. 11, 5213-5230.
\end{thebibliography}
%%%%%
%%%%%
\end{document}
