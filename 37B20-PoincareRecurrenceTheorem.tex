\documentclass[12pt]{article}
\usepackage{pmmeta}
\pmcanonicalname{PoincareRecurrenceTheorem}
\pmcreated{2013-03-22 14:29:50}
\pmmodified{2013-03-22 14:29:50}
\pmowner{Koro}{127}
\pmmodifier{Koro}{127}
\pmtitle{Poincar\'e recurrence theorem}
\pmrecord{6}{36033}
\pmprivacy{1}
\pmauthor{Koro}{127}
\pmtype{Theorem}
\pmcomment{trigger rebuild}
\pmclassification{msc}{37B20}
\pmclassification{msc}{37A05}

% this is the default PlanetMath preamble.  as your knowledge
% of TeX increases, you will probably want to edit this, but
% it should be fine as is for beginners.

% almost certainly you want these
\usepackage{amssymb}
\usepackage{amsmath}
\usepackage{amsfonts}
\usepackage{mathrsfs}

% used for TeXing text within eps files
%\usepackage{psfrag}
% need this for including graphics (\includegraphics)
%\usepackage{graphicx}
% for neatly defining theorems and propositions
\usepackage{amsthm}
% making logically defined graphics
%%%\usepackage{xypic}

% there are many more packages, add them here as you need them

\newtheorem{theorem}{Theorem}

% define commands here
\newcommand{\C}{\mathbb{C}}
\newcommand{\R}{\mathbb{R}}
\newcommand{\N}{\mathbb{N}}
\newcommand{\Z}{\mathbb{Z}}
\newcommand{\Per}{\operatorname{Per}}
\begin{document}
Let $(X,\mathscr{S},\mu)$ be a probability space and let $f\colon X\to X$
be a measure preserving transformation.

\begin{theorem} For any $E\in \mathscr{S}$,
the set of those points $x$ of $E$ such that $f^n(x)\notin E$ for all
$n>0$ has zero measure. That is, almost every point of $E$ returns to
$E$. In fact, almost every point returns infinitely often; i.e.
$$\mu\left(\{x\in E:\textnormal{ there exists } N \textnormal{
such that }f^n(x)\notin E \textnormal{ for all } n>N\}\right)=0.$$
\end{theorem}

The following is a topological version of this theorem:

\begin{theorem} If $X$ is a second countable Hausdorff space and
$\mathscr{S}$ contains the Borel sigma-algebra, then the
set of recurrent points of $f$ has full measure. That is, almost every
point is recurrent.
\end{theorem}
%%%%%
%%%%%
\end{document}
