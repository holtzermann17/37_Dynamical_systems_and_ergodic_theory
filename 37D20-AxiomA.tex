\documentclass[12pt]{article}
\usepackage{pmmeta}
\pmcanonicalname{AxiomA}
\pmcreated{2013-03-22 13:40:27}
\pmmodified{2013-03-22 13:40:27}
\pmowner{Koro}{127}
\pmmodifier{Koro}{127}
\pmtitle{Axiom A}
\pmrecord{7}{34340}
\pmprivacy{1}
\pmauthor{Koro}{127}
\pmtype{Definition}
\pmcomment{trigger rebuild}
\pmclassification{msc}{37D20}
\pmsynonym{hyperbolic diffeomorphism}{AxiomA}

% this is the default PlanetMath preamble.  as your knowledge
% of TeX increases, you will probably want to edit this, but
% it should be fine as is for beginners.

% almost certainly you want these
\usepackage{amssymb}
\usepackage{amsmath}
\usepackage{amsfonts}
\usepackage{mathrsfs}

% used for TeXing text within eps files
%\usepackage{psfrag}
% need this for including graphics (\includegraphics)
%\usepackage{graphicx}
% for neatly defining theorems and propositions
%\usepackage{amsthm}
% making logically defined graphics
%%%\usepackage{xypic}

% there are many more packages, add them here as you need them

% define commands here
\newcommand{\C}{\mathbb{C}}
\newcommand{\R}{\mathbb{R}}
\newcommand{\N}{\mathbb{N}}
\newcommand{\Z}{\mathbb{Z}}
\newcommand{\Per}{\operatorname{Per}}
\begin{document}
\PMlinkescapeword{satisfies}
Let $M$ be a smooth manifold. We say that a diffeomorphism $f\colon M\to M$ satisfies
(Smale's) \emph{Axiom A} (or that $f$ is an Axiom A diffeomorphism) if
\begin{enumerate}
\item the nonwandering set $\Omega(f)$ has a hyperbolic structure;
\item the set of periodic points of $f$ is dense in $\Omega(f)$: $\overline{\Per(f)} = \Omega(f)$.
\end{enumerate}
Sometimes, Axiom A diffeomorphisms are called hyperbolic diffeomorphisms, because the portion of $M$ where the ``interesting'' dynamics occur (namely, $\Omega(f)$) has a hyperbolic behaviour.
%%%%%
%%%%%
\end{document}
