\documentclass[12pt]{article}
\usepackage{pmmeta}
\pmcanonicalname{omegalimitSet}
\pmcreated{2013-03-22 13:39:37}
\pmmodified{2013-03-22 13:39:37}
\pmowner{Koro}{127}
\pmmodifier{Koro}{127}
\pmtitle{$\omega$-limit set}
\pmrecord{6}{34316}
\pmprivacy{1}
\pmauthor{Koro}{127}
\pmtype{Definition}
\pmcomment{trigger rebuild}
\pmclassification{msc}{37B99}
\pmsynonym{omega-limit set}{omegalimitSet}
\pmrelated{NonwanderingSet}
\pmdefines{$\alpha$-limit}
\pmdefines{alpha-limit}
\pmdefines{$\omega$-limit}
\pmdefines{omega-limit}

% this is the default PlanetMath preamble.  as your knowledge
% of TeX increases, you will probably want to edit this, but
% it should be fine as is for beginners.

% almost certainly you want these
\usepackage{amssymb}
\usepackage{amsmath}
\usepackage{amsfonts}
\usepackage{mathrsfs}

% used for TeXing text within eps files
%\usepackage{psfrag}
% need this for including graphics (\includegraphics)
%\usepackage{graphicx}
% for neatly defining theorems and propositions
%\usepackage{amsthm}
% making logically defined graphics
%%%\usepackage{xypic}

% there are many more packages, add them here as you need them

% define commands here
\newcommand{\C}{\mathbb{C}}
\newcommand{\R}{\mathbb{R}}
\newcommand{\N}{\mathbb{N}}
\newcommand{\Z}{\mathbb{Z}}
\begin{document}
Let $X$ be a metric space, and let $f:X\rightarrow X$ be a homeomorphism. 
The \emph{$\omega$-limit set} of $x\in X$, denoted by $\omega(x,f)$, is the set of cluster points of the forward orbit $\{f^n(x)\}_{n\in \N}$. 
Hence, $y\in \omega(x,f)$ if and only if there is a strictly increasing sequence of natural numbers $\{n_k\}_{k\in \N}$ such that $f^{n_k}(x)\rightarrow y$ as $k\rightarrow\infty$.

Another way to express this is
$$\omega(x,f) = \bigcap_{n\in \N} \overline{\{f^k(x): k>n\}}.$$

The \emph{$\alpha$-limit set} is defined in a similar fashion, but for the backward orbit; i.e. $\alpha(x,f)=\omega(x,f^{-1})$.

Both sets are $f$-invariant, and if $X$ is compact, they are compact and nonempty.

If $\varphi:\R\times X\to X$ is a continuous flow, the definition is similar:
$\omega(x,\varphi)$ consists of those elements $y$ of $X$ for which there exists a strictly increasing sequnece $\{t_n\}$ of real numbers such that $t_n\rightarrow \infty$ and $\varphi(x,t_n) \rightarrow y$ as $n\rightarrow\infty$.
Similarly, $\alpha(x,\varphi)$ is the $\omega$-limit set of the reversed flow (i.e. $\psi(x,t) = \phi(x,-t)$).
Again, these sets are invariant and if $X$ is compact they are compact and nonempty. Furthermore,
$$\omega(x,f) = \bigcap_{n\in \N}\overline{\{\varphi(x,t):t>n\}}.$$
%%%%%
%%%%%
\end{document}
