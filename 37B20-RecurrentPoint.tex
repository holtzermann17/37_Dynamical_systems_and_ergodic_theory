\documentclass[12pt]{article}
\usepackage{pmmeta}
\pmcanonicalname{RecurrentPoint}
\pmcreated{2013-03-22 14:29:53}
\pmmodified{2013-03-22 14:29:53}
\pmowner{Koro}{127}
\pmmodifier{Koro}{127}
\pmtitle{recurrent point}
\pmrecord{10}{36034}
\pmprivacy{1}
\pmauthor{Koro}{127}
\pmtype{Definition}
\pmcomment{trigger rebuild}
\pmclassification{msc}{37B20}
\pmrelated{NonwanderingSet}
\pmdefines{recurrent set}

% this is the default PlanetMath preamble.  as your knowledge
% of TeX increases, you will probably want to edit this, but
% it should be fine as is for beginners.

% almost certainly you want these
\usepackage{amssymb}
\usepackage{amsmath}
\usepackage{amsfonts}
\usepackage{mathrsfs}

% used for TeXing text within eps files
%\usepackage{psfrag}
% need this for including graphics (\includegraphics)
%\usepackage{graphicx}
% for neatly defining theorems and propositions
%\usepackage{amsthm}
% making logically defined graphics
%%%\usepackage{xypic}

% there are many more packages, add them here as you need them

% define commands here
\newcommand{\C}{\mathbb{C}}
\newcommand{\R}{\mathbb{R}}
\newcommand{\N}{\mathbb{N}}
\newcommand{\Z}{\mathbb{Z}}
\newcommand{\Per}{\operatorname{Per}}
\begin{document}
Let $X$ be a Hausdorff space and $f\colon X\to X$ a function. A point $x\in X$ is said to be \emph{recurrent} (for $f$) if $x\in \omega(x)$, i.e. if $x$ belongs to its $\omega$-\PMlinkname{limit}{OmegaLimitSet3} set. This means that for each neighborhood $U$ of $x$ there exists $n>0$ such that $f^n(x)\in U$.

The closure of the set of recurrent points of $f$ is often denoted $R(f)$ and is called the \emph{recurrent set} of $f$.

Every recurrent point is a nonwandering point, hence if $f$ is a homeomorphism and $X$ is compact, $R(f)$ is an invariant subset of $\Omega(f)$, which may be a proper subset.
%%%%%
%%%%%
\end{document}
