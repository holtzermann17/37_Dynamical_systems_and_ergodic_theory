\documentclass[12pt]{article}
\usepackage{pmmeta}
\pmcanonicalname{NocyclesCondition}
\pmcreated{2013-03-22 14:30:53}
\pmmodified{2013-03-22 14:30:53}
\pmowner{Koro}{127}
\pmmodifier{Koro}{127}
\pmtitle{no-cycles condition}
\pmrecord{5}{36054}
\pmprivacy{1}
\pmauthor{Koro}{127}
\pmtype{Definition}
\pmcomment{trigger rebuild}
\pmclassification{msc}{37-00}
\pmclassification{msc}{37C75}
\pmsynonym{no-cycles}{NocyclesCondition}
\pmsynonym{no-cycle}{NocyclesCondition}
\pmsynonym{no cycles condition}{NocyclesCondition}
\pmsynonym{cycle}{NocyclesCondition}

\endmetadata

% this is the default PlanetMath preamble.  as your knowledge
% of TeX increases, you will probably want to edit this, but
% it should be fine as is for beginners.

% almost certainly you want these
\usepackage{amssymb}
\usepackage{amsmath}
\usepackage{amsfonts}
\usepackage{mathrsfs}

% used for TeXing text within eps files
%\usepackage{psfrag}
% need this for including graphics (\includegraphics)
%\usepackage{graphicx}
% for neatly defining theorems and propositions
%\usepackage{amsthm}
% making logically defined graphics
%%%\usepackage{xypic}

% there are many more packages, add them here as you need them

% define commands here
\newcommand{\C}{\mathbb{C}}
\newcommand{\R}{\mathbb{R}}
\newcommand{\N}{\mathbb{N}}
\newcommand{\Z}{\mathbb{Z}}
\newcommand{\Per}{\operatorname{Per}}
\begin{document}
\PMlinkescapeword{satisfies}
Let $X$ be a metric space and let $f\colon X\to X$ be a homeomorphism.
Suppose $\mathcal F= \{\Lambda_1,\dots,\Lambda_k\}$ is a family of compact invariant sets for $f$. Define a relation $\rightarrow$ on $\mathcal{F}$ by $\Lambda_i \rightarrow \Lambda_j$ if $$W^u(\Lambda_i)\cap W^s(\Lambda_j) - \bigcup_{l=1}^k \Lambda_l \neq \emptyset,$$
that is, if the unstable set of $\Lambda_i$ intersects the stable set of $\Lambda_j$ outside the union of the $\Lambda_l$'s.

A cycle for $\mathcal F$ is a sequence $\{n_i:i=1,\dots,j\}$ such that 
$$\Lambda_{n_i}\rightarrow \Lambda_{n_{i+1}}$$
for $1\leq i<j$ and 
$$\Lambda_{n_j}\rightarrow \Lambda_{n_1}.$$
With some abuse of notation, we can write this as
$$\Lambda_{n_1}\rightarrow\Lambda_{n_2}\rightarrow\cdots\rightarrow\Lambda_{n_j}\rightarrow\Lambda_{n_1}.$$

If $\mathcal{F}$ has no cycles, then we say that it satisfies the \emph{no-cycles condition}.
%%%%%
%%%%%
\end{document}
