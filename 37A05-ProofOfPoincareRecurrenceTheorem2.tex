\documentclass[12pt]{article}
\usepackage{pmmeta}
\pmcanonicalname{ProofOfPoincareRecurrenceTheorem2}
\pmcreated{2013-03-22 14:29:58}
\pmmodified{2013-03-22 14:29:58}
\pmowner{Koro}{127}
\pmmodifier{Koro}{127}
\pmtitle{proof of Poincar\'e recurrence theorem 2}
\pmrecord{5}{36036}
\pmprivacy{1}
\pmauthor{Koro}{127}
\pmtype{Proof}
\pmcomment{trigger rebuild}
\pmclassification{msc}{37A05}
\pmclassification{msc}{37B20}

\endmetadata

% this is the default PlanetMath preamble.  as your knowledge
% of TeX increases, you will probably want to edit this, but
% it should be fine as is for beginners.

% almost certainly you want these
\usepackage{amssymb}
\usepackage{amsmath}
\usepackage{amsfonts}
\usepackage{mathrsfs}

% used for TeXing text within eps files
%\usepackage{psfrag}
% need this for including graphics (\includegraphics)
%\usepackage{graphicx}
% for neatly defining theorems and propositions
%\usepackage{amsthm}
% making logically defined graphics
%%%\usepackage{xypic}

% there are many more packages, add them here as you need them

% define commands here
\newcommand{\C}{\mathbb{C}}
\newcommand{\R}{\mathbb{R}}
\newcommand{\N}{\mathbb{N}}
\newcommand{\Z}{\mathbb{Z}}
\newcommand{\Per}{\operatorname{Per}}
\begin{document}
Let $\{U_n : n\in \N\}$ be a basis of open sets for $X$, and for each $n$ define
$$U_n' = \{x\in U_n : \forall n\geq 1,\, f^n(x) \notin U_n \}.$$
From theorem 1 we know that $\mu(U_n')=0$. Let $N=\bigcup_{n\in \N} U_n'.$
Then $\mu(N)=0$. We assert that if $x\in X-N$ then $x$ is recurrent. In fact,
given a neighborhood $U$ of $x$, there is a basic neighborhood $U_n$ such that $x\subset U_n\subset U$, and since $x\notin N$ we have that $x\in U_n-U_n'$ which by definition of $U_n'$ means that there exists $n\geq 1$ such that $f^n(x)\in U_n\subset U$; thus $x$ is recurrent. $\Box$
%%%%%
%%%%%
\end{document}
