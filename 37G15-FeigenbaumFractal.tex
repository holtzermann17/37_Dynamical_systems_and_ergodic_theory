\documentclass[12pt]{article}
\usepackage{pmmeta}
\pmcanonicalname{FeigenbaumFractal}
\pmcreated{2013-03-22 12:34:18}
\pmmodified{2013-03-22 12:34:18}
\pmowner{akrowne}{2}
\pmmodifier{akrowne}{2}
\pmtitle{Feigenbaum fractal}
\pmrecord{6}{32820}
\pmprivacy{1}
\pmauthor{akrowne}{2}
\pmtype{Definition}
\pmcomment{trigger rebuild}
\pmclassification{msc}{37G15}
\pmsynonym{Feigenbaum tree}{FeigenbaumFractal}
\pmdefines{logistic map}

\usepackage{amssymb}
\usepackage{amsmath}
\usepackage{amsfonts}

%\usepackage{psfrag}
\usepackage{graphicx}
%%%\usepackage{xypic}
\begin{document}
\PMlinkescapeword{cycle}
\PMlinkescapeword{image}
\PMlinkescapeword{graph}
A \emph{Feigenbaum fractal} is any bifurcation fractal produced by a period-doubling cascade.  The ``canonical'' Feigenbaum fractal is produced by the logistic map (a simple population model), 

$$ y' = \mu \cdot y  (1 - y) $$

where $\mu$ is varied smoothly along one dimension.  The logistic iteration either terminates in a cycle (set of repeating values) or behaves chaotically.  If one plots the points of this cycle versus the $\mu$-value, a graph like the following is produced:

\begin{center}
\includegraphics[scale=.8]{feigen.eps}
\end{center}

Note the distinct bifurcation (branching) points and the chaotic behavior as $\mu$ increases.

Many other iterations will generate this same type of plot, for example the iteration 

$$ p' = r \cdot \sin(\pi\cdot p) $$

One of the most amazing things about this class of fractals is that the bifurcation intervals are always described by Feigenbaum's constant.

Octave/Matlab Code to generate the above image is available \PMlinktofile{here}{octave_feigen.zip}.

\paragraph{References.}

\begin{itemize}
\item ``Quadratic Iteration, bifurcation, and chaos'': \PMlinkexternal{http://mathforum.org/advanced/robertd/bifurcation.html}{http://mathforum.org/advanced/robertd/bifurcation.html}
\item ``Bifurcation'': \PMlinkexternal{http://spanky.triumf.ca/www/fractint/bif_type.html}{http://spanky.triumf.ca/www/fractint/bif_type.html}
\item ``Feigenbaum's Constant'': \PMlinkexternal{http://fractals.iuta.u-bordeaux.fr/sci-faq/feigenbaum.html}{http://fractals.iuta.u-bordeaux.fr/sci-faq/feigenbaum.html}
\end{itemize}
%%%%%
%%%%%
\end{document}
