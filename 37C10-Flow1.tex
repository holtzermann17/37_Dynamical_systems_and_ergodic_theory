\documentclass[12pt]{article}
\usepackage{pmmeta}
\pmcanonicalname{Flow1}
\pmcreated{2013-03-22 13:12:34}
\pmmodified{2013-03-22 13:12:34}
\pmowner{Koro}{127}
\pmmodifier{Koro}{127}
\pmtitle{flow}
\pmrecord{8}{33673}
\pmprivacy{1}
\pmauthor{Koro}{127}
\pmtype{Definition}
\pmcomment{trigger rebuild}
\pmclassification{msc}{37C10}

% this is the default PlanetMath preamble.  as your knowledge
% of TeX increases, you will probably want to edit this, but
% it should be fine as is for beginners.

% almost certainly you want these
\usepackage{amssymb}
\usepackage{amsmath}
\usepackage{amsfonts}

% used for TeXing text within eps files
%\usepackage{psfrag}
% need this for including graphics (\includegraphics)
%\usepackage{graphicx}
% for neatly defining theorems and propositions
%\usepackage{amsthm}
% making logically defined graphics
%%%\usepackage{xypic}

% there are many more packages, add them here as you need them

% define commands here
\begin{document}
A \emph{flow} on a set $X$ is a group action of $(\mathbb{R},+)$ on $X$.

More explicitly, a flow is a function 
$\varphi:X\times \mathbb{R}\rightarrow X$
satisfying the following properties:
\begin{enumerate}
\item $\varphi(x,0) = x$ 
\item $\varphi(\varphi(x,t),s) = \varphi(x,s+t)$
\end{enumerate}
for all $s,t$ in $\mathbb{R}$ and $x\in X$.

The set $\mathcal{O}(x,\varphi) = \{\varphi(x,t):t\in\mathbb{R}\}$ is called the orbit of $x$ by $\varphi$.

Flows are usually required to be continuous or \PMlinkescapetext{even}  differentiable, when the space $X$ has some additional structure (e.g. when $X$ is a topological space or when $X = \mathbb{R}^n$.)

The most common examples of flows arise from describing the solutions of the autonomous ordinary differential equation 
\begin{equation}\label{eq1} y' = f(y),\;\;\; y(0)=x \end{equation}
as a function of the initial condition $x$, when the equation has existence and uniqueness of solutions.
That is, if (\ref{eq1}) has a unique solution $\psi_x:\mathbb{R}\rightarrow X$ for each $x\in X$, then $\varphi(x,t) = \psi_x(t)$ defines a flow.
%%%%%
%%%%%
\end{document}
