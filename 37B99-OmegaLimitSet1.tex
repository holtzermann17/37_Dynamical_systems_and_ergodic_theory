\documentclass[12pt]{article}
\usepackage{pmmeta}
\pmcanonicalname{OmegaLimitSet}
\pmcreated{2013-03-22 13:18:42}
\pmmodified{2013-03-22 13:18:42}
\pmowner{mathcam}{2727}
\pmmodifier{mathcam}{2727}
\pmtitle{omega limit set}
\pmrecord{5}{33818}
\pmprivacy{1}
\pmauthor{mathcam}{2727}
\pmtype{Definition}
\pmcomment{trigger rebuild}
\pmclassification{msc}{37B99}
\pmclassification{msc}{34C05}
\pmsynonym{$\omega$-limit set}{OmegaLimitSet}
\pmsynonym{$\alpha$-limit set}{OmegaLimitSet}
\pmrelated{LimitCycle}
\pmdefines{alpha limit set}

\endmetadata

% this is the default PlanetMath preamble.  as your knowledge
% of TeX increases, you will probably want to edit this, but
% it should be fine as is for beginners.

% almost certainly you want these
\usepackage{amssymb}
\usepackage{amsmath}
\usepackage{amsfonts}

% used for TeXing text within eps files
%\usepackage{psfrag}
% need this for including graphics (\includegraphics)
%\usepackage{graphicx}
% for neatly defining theorems and propositions
%\usepackage{amsthm}
% making logically defined graphics
%%%\usepackage{xypic}

% there are many more packages, add them here as you need them

% define commands here
\begin{document}
Let $\Phi(t,x)$ be the flow of the differential equation $x'=f(x)$, where $f\in C^k(M,\mathbb{R}^n)$, with $k\geq 1$ and $M$ an open subset of $\mathbb{R}^n$.
Consider $x\in M$. 

The omega limit set of $x$, denoted $\omega(x)$, is the set of points $y\in M$ such that there exists a sequence $t_n\to\infty$ with $\Phi(t_n,x)=y$.

Similarly, the alpha limit set of $x$, denoted $\alpha(x)$, is the set of points $y\in M$ such that there exists a sequence $t_n\to-\infty$ with $\Phi(t_n,x)=y$.

Note that the definition is the same for more general dynamical systems.
%%%%%
%%%%%
\end{document}
